%%%%%%%%%%%%%%%%%%%%%%%%%%%%%%%%%%%%%%%%%
% "ModernCV" CV and Cover Letter
% LaTeX Template
% Version 1.11 (19/6/14)
%
% This template has been downloaded from:
% http://www.LaTeXTemplates.com
%
% Original author:
% Xavier Danaux (xdanaux@gmail.com)
%
% License:
% CC BY-NC-SA 3.0 (http://creativecommons.org/licenses/by-nc-sa/3.0/)
%
% Important note:
% This template requires the moderncv.cls and .sty files to be in the same 
% directory as this .tex file. These files provide the resume style and themes 
% used for structuring the document.
%
%%%%%%%%%%%%%%%%%%%%%%%%%%%%%%%%%%%%%%%%%

%----------------------------------------------------------------------------------------
%	PACKAGES AND OTHER DOCUMENT CONFIGURATIONS
%----------------------------------------------------------------------------------------

\documentclass[11pt,a4paper,sans]{moderncv} % Font sizes: 10, 11, or 12; paper sizes: a4paper, letterpaper, a5paper, legalpaper, executivepaper or landscape; font families: sans or roman

\moderncvstyle{classic} % CV theme - options include: 'casual' (default), 'classic', 'oldstyle' and 'banking'
\moderncvcolor{black} % CV color - options include: 'blue' (default), 'orange', 'green', 'red', 'purple', 'grey' and 'black'

\usepackage{lipsum} % Used for inserting dummy 'Lorem ipsum' text into the template
\usepackage{amsmath}
\usepackage[english]{babel}

\usepackage[scale=.85]{geometry} % Reduce document margins
\setlength{\hintscolumnwidth}{2.5cm} % Uncomment to change the width of the dates column
%\setlength{\makecvtitlenamewidth}{10cm} % For the 'classic' style, uncomment to adjust the width of the space allocated to your name

%----------------------------------------------------------------------------------------
%	NAME AND CONTACT INFORMATION SECTION
%----------------------------------------------------------------------------------------

\firstname{Alejandro} % Your first name
\familyname{Reyes, PhD} % Your last name

% All information in this block is optional, comment out any lines you don't need
\title{\emph{Curriculum Vitae (\today)}}
%\address{2440 Massachusetts Avenue, Apt \#12}{Cambridge, 02140, USA}
%\mobile{+49 - 15772345859}
\email{alejandro.reyes.ds@gmail.com}
%\mobile{+1 - 617 961 2609}
%\fax{(000) 111 1113}
%
\extrainfo{
  \href{https://twitter.com/areyesq}{Twitter: @areyesq}\\
  \href{http://orcid.org/0000-0001-8717-6612}{ORCID: 0000-0001-8717-66}\\
  \href{https://scholar.google.com/citations?user=8QLuIWgAAAAJ}{Google Scholar: 8QLuIWgAAAAJ}\\
  \href{https://publons.com/a/389744/}{Publons: 389744}
}

%\email{\date{}}
%\homepage{staff.org.edu/~jsmith}{staff.org.edu/$\sim$jsmith} % The first argument is the url for the clickable link, the second argument is the url displayed in the template - this allows special characters to be displayed such as the tilde in this example
%\extrainfo{Birth date: February 12, 1989.}
%\photo[70pt][0.4pt]{pictures/picture} % The first bracket is the picture height, the second is the thickness of the frame around the picture (0pt for no frame)
%\quote{"A witty and playful quotation" - John Smith}


%----------------------------------------------------------------------------------------
\begin{document}
\makecvtitle % Print the CV title
\vspace{-1.2cm}
%----------------------------------------------------------------------------------------
%	EDUCATION SECTION
%----------------------------------------------------------------------------------------
\section{Summary}
My research is focused on developing analysis strategies that enable the translation of large amounts of data into biological knowledge. Broadly, I am interested in (1) understanding processes by which transcript isoforms contribute to cellular phenotypes and disease conditions and (2) integrating multi-omic data to unravel the molecular consequences of mutations in cancer. In order to ensure reproducibility of results and effective dissemination of code, I implement my analyses in documented  workflows, software packages and graphic interphases.

\section{Education}

\cventry{09/11 - 10/15}{Ph.D. in Biology}{\emph{European Molecular Biology Laboratory / University of Heidelberg}}{Heidelberg, Germany}{}{\emph{Summa cum laude.}}
\cventry{08/07 - 06/11}{B.Sc. in Genomic Sciences}{\emph{Autonomous National University of Mexico}}{Cuernavaca, Mexico}{}{With honours.}

%----------------------------------------------------------------------------------------
%	WORK EXPERIENCE SECTION
%----------------------------------------------------------------------------------------

\section{Research Experience}
\cventry{11/16 - today}{Postdoctoral Research Fellow}{}{Dana-Farber Cancer Institute and Harvard T.H. Chan School of Public Health}{Boston, USA}{
  \begin{itemize}
    \item Advisor: Prof. Rafael Irizarry.
    \item My current research is focused on the characterization of molecular phenotypes in cancer through the integration of multi-omic data. I am investigating how DNA methylation alters the three-dimensional structure of the genome and leads to activation of oncogenes. 
  \end{itemize}
}
\cventry{10/15 - 09/16}{Bridging Postdoctoral Fellow}{}{European Molecular Biology Laboratory}{Heidelberg, Germany}{
  \begin{itemize}
  \item Advisor: Dr. Wolfgang Huber.
  \item Analyzed data from the Genotype-Tissue Expression project to study transcript isoform dynamics across human tissues; Published a first author article.
  \end{itemize}
}
\cventry{09/11 - 10/15}{PhD Student}{}{European Molecular Biology Laboratory}{Heidelberg, Germany}{
  \begin{itemize}
  \item Advisor: Dr. Wolfgang Huber.
  \item Developed statistical software to analyze RNA-seq data; Used public datasets to investigate transcript
isoform dynamics across tissues and species; Engaged in collaborations with
experimentalist, participated in the design of research questions and experiments, and led the
computational analysis; Published 9 scientific articles, 6 as first author; Authored 3
R/Bioconductor packages; Trained PhD students and postdocs during yearly courses and workshops.
  \end{itemize}
}
\cventry{08/10 - 06/11}{Trainee}{}{European Molecular Biology Laboratory}{Heidelberg, Germany}{
  \begin{itemize}
  \item Advisor: Dr. Wolfgang Huber.
  \item Developed statistical methods to identify differences in transcript isoform regulation between different conditions.
  \end{itemize}
}
\cventry{06/10 - 08/10}{Trainee}{}{Weizmann Institute of Science}{Rehovot, Israel}{
  \begin{itemize}
  \item Advisor: Prof. Doron Lancet.
  \item Developed a computational pipeline to identify human genetic variants affecting olfactory receptors; Co-authored a peer-reviewed article.
  \end{itemize} }
\cventry{11/09 -- 06/10}{Undergraduate Research Assistant}{}{Autonomous National University of Mexico}{Cuernavaca, Mexico}{
  \begin{itemize}
  \item Advisors: Prof. Julio Collado-Vides and Prof. Enrique Morett.
  \item Evaluated methods to map transcription start sites in \emph{E. coli} using high-throughput sequencing data.
   \end{itemize}
}

\section{Honors}
\begin{itemize}
\item Mexican National System of Researchers (SNI I). \\
\end{itemize}
  
\section{Scientific publications}
\vspace{-.1cm}
\footnotesize{$^{\ast}$ Shared first authorship.} \hspace{.5cm}
\footnotesize{$^{\dagger}$ Shared last authorship.}
\vspace{.1cm}
\begin{itemize}
\item \textbf{\underline{A Reyes}}$^{\dagger}$ and W Huber$^{\dagger}$. Alternative start and termination sites of transcription drive most transcript isoform differences across human tissues. \textit{Nucleic Acids Research}, 2017. \href{https://doi.org/10.1093/nar/gkx1165}{doi: 10.1093/nar/gkx1165}
\item M Ruiz-Velasco, ..., \textbf{\underline{A Reyes}}, ..., JB Zaugg. CTCF-mediated chromatin loops between promoter and gene body regulate alternative splicing across individuals. \textit{Cell Systems}, 2017. \href{http://dx.doi.org/10.1016/j.cels.2017.10.018}{doi: 10.1016/j.cels.2017.10.018}
\item MM Parker, ..., \textbf{\underline{A Reyes}}, ..., PJ Casaldi. RNA sequencing identifies novel non-coding RNA and exon-specific effects associated with cigarette smoking. \textit{BMC Medical Genomics}, 2017. \href{https://doi.org/10.1186/s12920-017-0295-9}{doi: 10.1186/s12920-017-0295-9}
\item R Scognamiglio, ..., \textbf{\underline{A Reyes}}, ..., A Trumpp. Myc depletion induces a pluripotent dormant state mimicking diapause. \textit{Cell}, 2016. \href{https://doi.org/10.1016/j.cell.2015.12.033}{doi: 10.1016/j.cell.2015.12.033}
\item P Brennecke$^{\ast}$, \textbf{\underline{A Reyes}}$^{\ast}$, S Pinto$^{\ast}$, K Rattay$^{\ast}$, ..., W Huber$^{\dagger}$, B Kyewski$^{\dagger}$ and LM Steinmetz$^{\dagger}$. Single-cell transcriptome analysis reveals coordinated ectopic gene-expression patterns in medullary thymic epithelial cells. \textit{Nature Immunology}, 2015. \href{https://doi.org/10.1038/ni.3246}{doi: 10.1038/ni.3246}
\item W Huber, ..., \textbf{\underline{A Reyes}}, ..., M Morgan. Orchestrating high-throughput genomic analysis with Bioconductor. \textit{Nature Methods}, 2015. \href{https://doi.org/10.1038/ni.3246}{doi: 10.1038/ni.3246}
\item D Klimmeck$^{\ast}$, N Cabezas-Wallscheid$^{\ast}$, \textbf{\underline{A Reyes}}$^{\ast}$, ..., W Huber$^{\dagger}$ and A Trumpp$^{\dagger}$. Transcriptome-wide profiling and posttranscriptional analysis of hematopoietic stem/progenitor cell differentiation toward myeloid commitment. \textit{Stem Cell Reports}, 2014. \href{https://doi.org/10.1016/j.stemcr.2014.08.012}{doi: 10.1016/j.stemcr.2014.08.012}
\item N Cabezas-Wallscheid$^{\ast}$, D Klimmeck$^{\ast}$, J Hansson$^{\ast}$, DB Lipka$^{\ast}$, \textbf{\underline{A Reyes}}$^{\ast}$, ..., W Huber$^{\dagger}$, MD Milsom$^{\dagger}$, C Plass$^{\dagger}$, J Krijgsveld$^{\dagger}$ and A Trumpp$^{\dagger}$. Identification of regulatory networks in HSCs and their immediate progeny via integrated proteome, transcriptome, and DNA methylome analysis. \textit{Cell Stem Cell}, 2014. \href{https://doi.org/10.1016/j.stem.2014.07.005}{doi: 10.1016/j.stem.2014.07.005}
\item \textbf{\underline{A Reyes}}, ..., W Huber. Mutated SF3B1 is associated with transcript isoform changes of the genes UQCC and RPL31 both in CLLs and uveal melanomas. \textit{bioRxiv}, 2013. \href{https://doi.org/10.1101/000992}{doi: 10.1101/000992}
\item \textbf{\underline{A Reyes}}$^{\ast}$, S Anders$^{\ast}$, ..., W Huber. Drift and conservation of differential exon usage across tissues in primate species. \textit{PNAS}, 2013. \href{https://doi.org/10.1073/pnas.1307202110}{doi: 10.1073/pnas.1307202110}
\item K Zarnack$^{\ast}$, J K\"{o}nig$^{\ast}$, ..., \textbf{\underline{A Reyes}}, ..., NM Luscombe$^{\dagger}$ and J Ule$^{\dagger}$. Direct competition between hnRNP C and U2AF65 protects the transcriptome from the uncontrolled exonization of Alu elements. \textit{Cell}, 2013. \href{https://doi.org/10.1016/j.cell.2012.12.023}{doi: 10.1016/j.cell.2012.12.023}
\item T Olender, ..., \textbf{\underline{A Reyes}}, ..., D Lancet. Personal receptor repertoires: olfaction as a model. \textit{BMC Genomics}, 2012. \href{https://doi.org/10.1186/1471-2164-13-414}{doi: 10.1186/1471-2164-13-414}
\item S Anders$^{\ast}$, \textbf{\underline{A Reyes}}$^{\ast}$ and W Huber. Detecting differential usage of exons from RNA-seq data. \textit{Genome Research}, 2012. \href{https://doi.org/10.1101/gr.133744.111}{doi: 10.1101/gr.133744.111}
\end{itemize}
\vspace{-.13cm}
\section{Software development}
\vspace{-.1cm}
\begin{itemize}
\item \href{http://www.bioconductor.org/packages/release/bioc/html/DEXSeq.html}{DEXSeq}: Inference of differential exon usage from RNA-seq data. \textit{R/Bioconductor}.
\item \href{http://bioconductor.org/packages/release/data/experiment/html/pasilla.html}{pasilla}: Package with count data of a pasilla knock-down RNA-seq experiment. \textit{R/Bioconductor}.
\item \href{http://bioconductor.org/packages/release/data/experiment/html/Single.mTEC.Transcriptomes.html}{Single.mTEC.Transcriptomes}: Transcriptome data and analysis of mouse mTECs. \textit{R/Bioconductor}.
\item \href{http://www.bioconductor.org/packages/release/bioc/html/SummarizedBenchmark.html}{SummarizedBenchmark}: Inference of differential exon usage from RNA-seq data. \textit{R/Bioconductor}.
\end{itemize}
\vspace{-.2cm}
\section{Presentations and Posters}
\vspace{-.1cm}
\textbf{Invited talks}
\begin{itemize}
\item Blue Seminar. European Molecular Biology Laboratory, Heidelberg, Germany, 2017. 
\item Evolution of Biological Traits. Center for Advanced Studies (LMU), Munich, Germany, 2017.
\item Seminarios de Investigaci\'on. Universidad del Valle de Atemajac, Queretaro, Mexico, 2017.
\item Genomeeting. National Institute of Genomic Medicine, Mexico city, Mexico, 2016. 
\item 15th Annual BCI-McGill Workshop. Bellairs Research Institute, Holetown, Barbados, 2016.
\item C1omics Workshop. Manchester Cancer Research Centre, Manchester, UK, 2015.
\item Interpretation of Next Generation Sequencing Data Workshop. Heidelberg University, Heidelberg, Germany, 2015. 
\item RADIANT General Meeting. Telethon Institute of Genetics and Medicine, Pozzuoli, Italy, 2015.
\item ``Manejo Inteligente de Datos e Informaci\'on''. Mexican Institute of Transportation, Queretaro, Mexico, 2014.
\item European Conference on Computational Biology \emph{RADIANT} Workshop. Strasbourg, France, 2014.
\item Statistical Analysis of RNA-seq Data. Pasteur Institut, Paris, France, 2013.
\item BioC Conference. Fred Hutchison Cancer Research Center, Seattle, USA, 2013.
\end{itemize}
\vspace{.1cm}
\textbf{Selected talks}
\begin{itemize}
\item The Biology of Genomes. Cold Spring Harbor Laboratory, Cold Spring Harbor, USA, 2014. 
\end{itemize}
\vspace{.1cm}
\textbf{Poster presentations}
\begin{itemize}
\item Single-cell Genomics Conference. Hubrecht Institute, Utrecht, Netherlands, 2015.
\item BioC Conference. Fred Hutchison Cancer Research Center, Seattle, USA, 2012.
\item Cancer Genomics Conference. European Molecular Biology Laboratory, Heidelberg, Germany, 2012.
\end{itemize}

%------------------------------------------------
\vspace{-.2cm}
\section{Teaching}
\vspace{-.1cm}
\textbf{Mentor and lecturer}
\begin{itemize}
\item Workshop on Transcriptomics, Harvard University. Cambridge, USA, 2017. 
\item UNAM's II Summer School in Bioinformatics. Juriquilla, Mexico, 2017. 
\item Replicathon2017: Consistency of Large Pharmacogenomic Studies. Bayam\'on, Puerto Rico, 2017.
\item Genomeeting Workshop: Analysis of RNA-seq data. Mexico city, Mexico, 2016.
\item Statistics and Computing in Genome Data Science. Bressanone, Italy, 2015.
\item Data Analysis for Genome Biology. Bressanone, Italy, 2014.
\item Computational Statistics for Genome Biology. Bressanone, Italy, 2013.
\item BioC Conference. Seattle, USA, 2012.
\end{itemize}
\vspace{.1cm}
\textbf{Teaching assistant}
\begin{itemize}
\item Introduction to Data Science: BST260. Harvard T.H. Chan School of Public Health, Boston, USA, 2017. 
\item Advanced topics in Evolutionary Genomics. {\v C}zern\'y  Krumov, Czech Republic, 2013.
\item Computational Statistics for Genome Biology. Bressanone, Italy, 2012. 
\item Computational Statistics for Genome Biology. Bressanone, Italy, 2011. 
\item Introduction to R/Bioinformatics. Autonomous National University of Mexico. Cuernavaca, Mexico, 2010.
\end{itemize}

%\section{Skills}
%\cvitem{Programming languages}{R/Bioconductor, python, perl, C, \LaTeX, mySQL}
%\cvitem{Languages}{Spanish (native), English (advanced), Italian (acquired), German (basic)}
%\cvitem{Other}{Piano}

%\section{References}
%\begin{itemize}
%\item Dr. Wolfgang Huber (whuber@embl.de).
%Principal Investigator and Senior Scientist at the European Molecular Biology Laboratory.
%\item Prof. Lars M. Steinmetz (larsms@stanford.edu).
%Professor of Genetics at Stanford University. Co-Director of the Stanford Genome Technology Center.
%Principal Investigator, Associate Head of Genome Biology and Senior Scientist at the European Molecular Biology Laboratory.
%\end{itemize}

\end{document}
